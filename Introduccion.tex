\chapter{Introducción}
En la antigua Grecia, a parte de las rectas, planos y ciertos espacios en dos dimensiones, según registros basados en \cite{flores_1970}, lo cual señala que los griegos tendrían interés en las curvas obtenidas de un cono, actualmente conocidas como secciones cónicas, en un intento por resolver el problema de "Delos" tenemos a Menecmo ($320$ a. C.) en su intento de resolverlo geométricamente. Uno de los procesos más importantes de la matemática, se centra en resolver problemas, dichos problemas se han extendido más allá de un estudio particular de una entrada, Resolver problemas que implicaba áreas y volúmenes, estaban más allá de un plano y una recta, lo cual introdujo nuevas nociones matemáticas como curvas y superficies. Los pioneros dentro del cálculo fue Isaac Newton ($1642-1727$), basado en \cite{zill1} menciona, además, que otro de los matemáticos en centrarse en el cálculo es Gottfried Wilhelm Leibniz ($1646-1711$), el calculo a permitido resolver diferentes problemas, dentro de diferentes ramas, dentro de la física a permitido estudiar el comportamiento de diferentes fenómenos, para la química aportado el estudio avanzado de la materia, el calculo es parte del crecimiento humano. Sin embargo, aún tiene sus misterios podemos mencionar a que Arquímedes no pudo resolver el problema fundamental del cálculo diferencial. Esta nueva perspectiva de las matemáticas, han permitido el nacimiento de nuevas teorías y soluciones a problemas ya propuestos.