
    \section{Introducción}
    Las demostraciones, son una parte esencial dentro de las matemáticas; existen diferentes métodos para poder cumplir este objetivo, en este documentos trataremos 
    acerca del método de reducción al absurdo, dicho método dentro de las matemáticas se vio por primera en la proposición 11 de elementos de la matemática de Euclides 
    de Alejandría (\ref{euclidesSeisLibrosPrimeros1576}) y Arquímedes de Siracusa (\ref{parra}) son dos ejemplos muy tempranos. Sin embargo, las dos proposiciones más comunes, 
    que demuestran a través de este método son: “existen infinitos números primos” y “la irracionalidad de los números”; el razonamiento general de la demostración por reducción 
    al absurdo se basa en negar la tesis y llegar a una contradicción; a lo cual se genera una pregunta, ¿es suficiente demostrar por reducción al absurdo?, Kurl Godel en su obra 
    “Sobre proposiciones formalmente indecibles de los 'Principia Mathematica' y sistemas afines” (\cite{bauer}) , sostiene que existen proposiciones que el 
    razonamiento matemático no es capaz de demostrar, además, mencionar la famosa paradoja de Epimènides, famosamente conocida como la paradoja del mentiroso. Es importante 
    mencionar que el valor que este método tienen dentro de la formación académica de los matemáticos y la construcción de teorías de la misma recae a su viabilidad en 
    demostraciones sobre la unicidad y la existencia.
